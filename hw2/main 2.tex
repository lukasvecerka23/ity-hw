\documentclass[hidelinks ,twoside, 11pt ,twocolumn, a4paper]{article}
\usepackage[left=1.4cm, top=2.3cm, text={18.2cm, 25.2cm}]{geometry} 
\usepackage[utf8]{inputenc}
\usepackage{changepage}
\usepackage[czech]{babel}
\usepackage[T1]{fontenc}
\usepackage{hyperref}
\usepackage{color}
\usepackage{alltt}
\usepackage[IL2]{fontenc}
%\usepackage{mathptmx}
\usepackage{times}
\usepackage{amsthm}
\usepackage{amsmath}
\usepackage{amsfonts}
\newtheorem{definition}{Definice}
\newtheorem{theorem}{Věta}
\usepackage{tipa}

\begin{document}

\begin{titlepage}
        \thispagestyle{empty}
        \begin{center}
            \Huge
            \textsc{Vysoké učení technické v Brně\\}
            \vspace{0.5em}
            \huge
            \textsc{Fakulta informačních technologií\\}
            \vspace{\stretch{0.382}}
            \LARGE
            Typografie a publikování\,--\,2. projekt \\
            \vspace{0.4em}
            Sazba dokumentů a matematických výrazů\\
            \vspace{\stretch{0.618}}
        \end{center}
        {\Large 2023 \hfill Veronika Nevařilová (xnevar00)}
    \end{titlepage}

\clearpage
\pagenumbering{arabic}
\section*{Úvod}
V~této úloze si vyzkoušíme sazbu titulní strany, matematických vzorců, prostředí a~dalších textových struktur obvyklých pro technicky zaměřené texty\,--\,například Definice~\ref{definice1} nebo rovnice~\eqref{rovnice3} na straně~\pageref{rovnice3}. Pro vytvoření těchto odkazů používáme kombinace příkazů \verb|\label|, \verb|\ref|, \verb|\eqref| a~\verb|\pageref|. Před odkazy patří nezlomitelná mezera. Pro zvýrazňování textu jsou zde několikrát použity příkazy \verb|\verb| a~\verb|\emph|. 

Na titulní straně je použito prostředí \verb|titlepage| a~sázení nadpisu podle optického středu s~využitím \textit{přesného} zlatého řezu. Tento postup byl probírán na přednášce. Dále jsou na titulní straně použity čtyři různé velikosti písma a~mezi dvojicemi řádků textu je použito odřádkování se zadanou relativní velikostí 0,5\,em a~0,4\,em\footnote{Nezapomeňte použít správný typ mezery mezi číslem a~jednotkou.}.

\section{Matematický text}
V této sekci se podíváme na sázení matematických symbolů a~výrazů v~plynulém textu pomocí prostředí \verb|math|. Definice a~věty sázíme pomocí příkazu \verb|\newtheorem| s~využitím balíku \verb|amsthm|. Někdy je vhodné použít konstrukci\ \verb|${}$|\ nebo\ \verb|\mbox{}|, která říká, že (matematický) text nemá být zalomen.

\begin{definition}
\label{definice1}
{\upshape
Zásobníkový automat \textit{(ZA) je definován jako sedmice tvaru} 
$A = (Q, \Sigma, \Gamma, \delta, q_0, Z_0, F)$, kde: 

\begin{itemize}

\item $Q$ \textit{je konečná množina} vnitřních (řídicích) stavů,
\item $\Sigma$ \textit{je konečná} vstupní abeceda,
\item $\Gamma$ \textit{je konečná} zásobníková abeceda,
\item $\delta$ je přechodová funkce $Q \times (\Sigma \cup \left\{ \epsilon\right\}) \times \Gamma \rightarrow 2^{Q \times \Gamma^*}$,
\item $q_0 \in Q$ je počáteční stav, $Z_0 \in \Gamma$ je startovací symbol zásobníku \textit{a~$F \subseteq Q$ je množina} koncových stavů.
\end{itemize}

Nechť $P = (Q, \Sigma, \Gamma, \delta, q_0, Z_0, F)$ je ZA. \textit{Konfigurací} nazveme trojici $(q, w, \alpha) \in Q \times \Sigma^* \times \Gamma^*$, kde $q$ je aktuální stav vnitřního řízení, $w$ je dosud nezpracovaná část vstupního řetězce a~$\alpha = Z_{i_{1}} Z_{i_{2}} \dots Z_{i_{k}}$ je obsah zásobníku.

}
\end{definition}

\subsection{Podsekce obsahující definici a~větu}
\begin{definition}
    {\upshape Řetězec $w$ nad abecedou $\Sigma$ je přijat ZA $A$~\textit{jestliže $(q_0, w, Z_0) \overset{*}{\underset{A}{\vdash}} (q_F, \epsilon, \gamma)$ pro nějaké $\gamma \in \Gamma^*$ a $q_F \in F$. Množina $L(A) = \{ w\ |\ w \text{\textit{ je přijat ZA }}A\} \subseteq \Sigma^*$ je} jazyk přijímaný ZA \textit{A}.
    }
\end{definition}

\begin{theorem}
Třída jazyků, které jsou přijímány ZA, odpovídá {\upshape bezkontextovým jazykům.}
\end{theorem}

\section{Rovnice}
Složitější matematické formulace sázíme mimo plynulý text pomocí prostředí \verb|displaymath|. Lze umístit i~několik výrazů na jeden řádek, ale pak je třeba tyto vhodně oddělit, například příkazem \verb|\quad|. 
\begin{displaymath}
1^{2^3} \neq \Delta^1_{\Delta^2_{\Delta^3}}\quad
y^{11}_{22} - \sqrt[9]{x+\sqrt[7]{y}}\quad
 x > y_1 \le y^2
\end{displaymath}
V~rovnici~\eqref{rovnice2} jsou využity tři typy závorek s~různou \textit{explicitně} definovanou velikostí. Také nepřehlédněte, že nasledující tři rovnice mají zarovnaná rovnítka, a~použijte k~tomuto účelu vhodné prostředí.
\begin{eqnarray}
-\cos^2\beta & = & \frac{\frac{\frac{1}{x}+\frac{1}{3}}{y}+1000}{\prod\limits _{j=2}^8 q_j}\\
\label{rovnice2}
\bigg(\Big\{b\star\big[3\div4\big]\circ a\Big\}^{\frac{2}{3}}\bigg) & = & \log _{10} x\\
\label{rovnice3} 
\int_a^b f(x)\,\mathrm{d}x & = & \int_c^d f(y)\,\mathrm{d}y
\end{eqnarray}
V~této větě vidíme, jak vypadá implicitní vysázení limity $\lim_{m \to \infty} f(m)$ v normálním odstavci textu. Podobně je to~i~s~dalšími symboly jako $\bigcup_{N\in\mathcal{M}} N$ či $\sum_{i=1}^m x_i^2$. S vynucením méně úsporné sazby příkazem \verb|\limits| budou vzorce vysázeny v podobě $\lim\limits_{m \to \infty} f(m)$ a~$\sum\limits_{i=1}^m x_i^4$.

\section{Matice}
Pro sázení matic se velmi často používá prostředí \verb|array| a~závorky (\verb|\left|, \verb|\right|).

\[\mathbf{B} = \left| \begin{array}{cccc}
    b_{11} & b_{12} & \cdots & b_{1n}\\
    b_{21} & b_{22} & \cdots & b_{2n} \\
    \vdots & \vdots & \ddots & \vdots \\
    b_{m1} & b_{m2} & \cdots & b_{mn} \\
    \end{array} \right| = 
    \left| \begin{array}{cc}
    t & u \\
    v & w \\
    \end{array} \right| = 
    tw-uv\]
\[\mathbb{X} = \mathbf{Y} \Longleftrightarrow {\left[\begin{array}{ccc}
    \null & \Omega+\Delta & \hat{\psi}\\
    \vec{\pi} & \omega & \null
    \end{array} \right]} \ne 42\]

Prostředí \verb|array| lze úspěšně využít i~jinde, například na
pravé straně následující rovnice. Kombinační číslo na levé
straně vysázejte pomocí příkazu \verb|\binom|.
\[
\binom{n}{k} = \left\{
\begin{array}{c l}
0 & \text{pro } k < 0 \\
\frac{n!}{k!(n-k)!} & \text{pro } 0 \le k \le n\\
0 & \text{pro } k > 0 \\
\end{array} \right.
\]


\end{document}
