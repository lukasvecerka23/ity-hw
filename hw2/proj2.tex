\documentclass[hidelinks, 11pt, a4paper, twocolumn]{article}
\usepackage[czech]{babel}
\usepackage[left=1.4cm,text={18.2cm, 25.2cm},top=2.3cm]{geometry}
\usepackage[utf8]{inputenc}
\usepackage{hyperref}
\usepackage{xcolor}
\usepackage[IL2]{fontenc} 
\usepackage{changepage}
\usepackage{times}
\usepackage{amsmath}
\usepackage{amsthm}
\usepackage{amsfonts}
\usepackage{stackrel}


\theoremstyle{definition}
\newtheorem{definice}{Definice}

\theoremstyle{definition}
\newtheorem{veta}{Věta}

\begin{document}

\begin{titlepage}
    \begin{center}
        \textsc{\Huge Vysoké učení technické v Brně\\[0.5em]
        \huge Fakulta informačních technologií}\\
        \vspace{\stretch{0.382}}
        {\LARGE Typografie a publikování\,--\,2. projekt\\[0.4em]
        Sazba dokumentů a matematických výrazů}\\
        \vspace{\stretch{0.618}}
    \end{center}
        {\Large 2023 \hfill Lukáš Večerka (xvecer30)}
\end{titlepage}

\newpage

\section*{Úvod}
\label{strana1}
V této úloze si vyzkoušíme sazbu titulní strany, matematických vzorců,
 prostředí a dalších textových struktur obvyklých pro technicky zaměřené
  texty\,--\,například Definice~\ref{definice1} nebo rovnice~\eqref{rovnice3} na straně~\pageref{strana1}. Pro vytvoření těchto odkazů
   používáme kombinace příkazů \verb|\label|, \verb|\ref|, \verb|\eqref| a \verb|\pageref|.
    Před odkazy patří nezlomitelná mezera. Pro zvýrazňování textu jsou zde několikrát
     použity příkazy \verb|\verb| a \verb|\emph|.
  
Na titulní straně je použito prostředí \texttt{titlepage} a
      sázení nadpisu podle optického středu s využitím \emph{přesného} zlatého řezu. Tento
       postup byl probírán na přednášce. Dále jsou na titulní straně použity čtyři různé
        velikosti písma a mezi dvojicemi řádků textu je použito odřádkování se zadanou relativní
         velikostí 0,5\,em a 0,4\,em\footnote[1]{Nezapomeňte použít správný typ mezery mezi číslem a jednotkou.}.

\section{Matematický text}
V této sekci se podíváme na sázení matematických symbolů a výrazů v plynulém textu pomocí prostředí \texttt{math}.
 Definice a věty sázíme pomocí příkazu \verb|\newtheorem| s využitím balíku \texttt{amsthm}. Někdy je vhodné použít konstrukci
  \verb|${}$| nebo \verb|\mbox{}|, která říká, že (matematický) text nemá být zalomen.

\begin{definice}
\label{definice1}
Zásobníkový automat
 \emph{(ZA) je definován jako sedmice tvaru $A=(Q, \Sigma, \Gamma, \delta, q_{0}, Z_{0}, F) $, kde}:
 \begin{itemize}
  \item \emph{$Q$ je konečná množina} vnitřních (řídicích) stavů,
   \item \emph{$\Sigma$ je konečná} vstupní abeceda,
   \item \emph{$\Gamma$ je konečná} zásobníková abeceda,
   \item \emph{$\delta$ je} přechodová funkce $Q \times (\Sigma \cup \{\epsilon\}) \times \Gamma \to 2^{Q \times \Gamma^\ast} $,
   \item $q_{0} \in Q$ \emph{je} počáteční stav, $Z_{0} \in \Gamma$ \emph{je} startovací symbol zásobníku \emph{a $F \subseteq Q$ je množina} koncových
     stavů.
\end{itemize}

    Nechť $P=(Q, \Sigma, \Gamma, \delta, q_{0}, Z_{0}, F)$ je ZA. \emph{Konfigurací} nazveme trojici $(q,w,\alpha) \in Q \times \Sigma^\ast \times \Gamma^\ast$,
     kde $q$ je aktuální stav vnitřního řízení, $w$ je dosud nezpracovaná část
      vstupního řetězce a $\alpha = Z_{i_1}Z_{i_2} \dots Z_{i_k}$ je obsah zásobníku.

\end{definice}

\subsection{Podsekce obsahující definici a větu}

\begin{definice}
\label{definice2}
Řetězec $w$ nad abecedou $\Sigma$ je přijat ZA~$A$~\emph{jestliže $(q_0, w, Z_0)\stackrel[A]{\ast}{\vdash}(q_F, \epsilon, \gamma) $ pro nějaké $\gamma \in \Gamma^\ast \text{ a } q_F \in F$. Množina $L(A) = \{w \mid w \text{ \textit{je přijat ZA }}A\} \subseteq \Sigma^\ast$ je} jazyk přijímaný ZA $A$.
\end{definice}

\begin{veta}
\label{veta1}
\emph{Třída jazyků, které jsou přijímány ZA, odpovídá} bezkontextovým jazykům.
\end{veta}

\section{Rovnice}
Složitější matematické formulace sázíme mimo plynulý text pomocí prostředí \texttt{displaymath}. Lze umístit i několik výrazů na jeden
 řádek, ale pak je třeba tyto vhodně oddělit, například příkazem \verb|\quad|.
 \begin{displaymath}
    1^{2^3} \neq \Delta^1_{\Delta^2_{\Delta^3}}
    \quad y^{11}_{22} - \sqrt[9]{x + \sqrt[7]{y}}
    \quad x > y_1 \leq y^2   
 \end{displaymath}
 V rovnici (\ref{rovnice2}) jsou využity tři typy závorek s různou
  \emph{explicitně} definovanou velikostí. Také nepřehlédněte, že nasledující tři rovnice mají zarovnaná rovnítka, a použijte k~tomuto
   účelu vhodné prostředí.
   \begin{eqnarray}
        \label{rovnice1}
        -\cos^2 \beta & = & \frac{\frac{\frac{1}{x}+\frac{1}{3}}{y}+1000}{\prod\limits _{j=2}^8 q_j}\\
        \label{rovnice2}
        \biggl(\Bigl\{b \star \bigl[3 \div 4\bigr] \circ a \Bigr\}^{\frac{2}{3}}\biggr) & = & \log_{10}x\\
        \label{rovnice3}
        \int_a^b f(x)\,\mathrm{d}x & = & \int_c^d f(y)\,\mathrm{d}y
   \end{eqnarray}
   V této větě vidíme, jak vypadá implicitní vysázení limity $\lim_{m\to\infty}f(m)$ v normálním odstavci textu. Podobně je to i
    s dalšími symboly jako $\bigcup_{N \in \mathcal{M}}N $ či $\sum_{i=1}^{m} x^2_i $.
     S vynucením méně úsporné sazby příkazem \verb|\limits| budou vzorce vysázeny v podobě $\lim\limits _{m \to \infty} f(m)$ a $\sum\limits _{i=1}^{m} x^4_i$.

\section{Matice}
Pro sázení matic se velmi často používá prostředí \texttt{array} a závorky (\verb|\left|, \verb|\right|).
$$
\mathbf{B} = \left| \begin{array}{cccc}
    b_{11} & b_{12} & \cdots & b_{1n}\\ 
    b_{21} & b_{22} & \cdots & b_{2n} \\
    \vdots & \vdots & \ddots & \vdots \\
    b_{m1} & b_{m2} & \cdots & b_{mn} 
\end{array}
\right| = \left|\begin{array}{cc}
    t & u \\
    v & w 
\end{array}
\right| = tw - uv
$$
 
$$\mathbb{X}=\mathbf{Y}\Longleftrightarrow
\left[\begin{array}{ccc}
    &\Omega + \Delta & \hat{\psi} \\
    \vec{\pi} & \omega & 
\end{array}
\right] \neq 42$$

Prostředí \texttt{array} lze úspěšně využít i jinde,
 například na pravé straně následující rovnice. Kombinační číslo na levé straně vysázejte pomocí příkazu \verb|\binom|.

$$ \binom{n}{k} = \left\{
\begin{array}{c l}
0 & \text{pro } k < 0 \\
\frac{n!}{k!(n-k)!} & \text{pro } 0 \leq k \leq n \\
0 & \text{pro } k > 0
\end{array} \right. $$

\end{document}