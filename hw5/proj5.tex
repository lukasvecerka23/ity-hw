\documentclass{beamer}
\usepackage[czech]{babel}
\usepackage[utf8]{inputenc}
\usepackage[czech, ruled, vlined]{algorithm2e}
\usepackage{tikz}
\usepackage{hyperref}
\usetikzlibrary {arrows.meta}

\usetheme[width=2cm]{PaloAlto}
\setbeamertemplate{itemize item}{\textbullet}

\addtobeamertemplate{footline}{}{ \hfill   \usebeamerfont{footline}%
    \insertframenumber/\inserttotalframenumber\hspace{.3cm} }

\title[Tarjanův algoritmus]{\huge Tarjanův algoritmus}
\author[Lukáš Večerka \\ xvecer30]{Lukáš Večerka \\ xvecer30}
\date{\today}

\begin{document}
\begin{frame}
    \maketitle
\end{frame}

\begin{frame}{Obsah}
\tableofcontents
\end{frame}

\section{Úvod}
\begin{frame}{Úvod}
    \begin{itemize}
        \item Algoritmus sloužící k vyhledávání silných komponent v orientovaném grafu
        \item Vymyslel ho americký informatik Robert Tarjan v roce 1972
        \item Silná komponenta je podgraf, ve kterém je každý uzel dosažitelný z každého jiného uzlu
        \item Využívá se při optimalizaci programů, analýze dosažitelnosti a teorii grafů
    \end{itemize}
\end{frame}

\section{Princip}
\begin{frame}{Princip}
    \begin{itemize}
        \item Využívá zásobník a dvě pole
        \item Průchod grafem pomocí DFS, začínající u libovolného uzlu
        \item Pro každý uzel se ukládá čas objevení a nejmenší čas objevení uzlu
        \item Při zpětném volání se aktualizují nejmenší časy objevení uzlů rodičů
        \item Pomocí nejmenších časů se provádí identifikace silných komponent
        \item Pokud je čas objevení uzlu roven nejmenšímu času objevení uzlu, je uzel kořenem silné komponenty
    \end{itemize}
\end{frame}

\SetAlFnt{\small}
\section{Algoritmus 1}
\begin{frame}{Algoritmus 1. část}
            \begin{algorithm}[H]
                \DontPrintSemicolon
                \SetAlgoNoLine
                \SetAlgoNoEnd
                \KwIn{\textit{Graf} $G = (V, E)$}
                \KwOut{\textit{Množina silných komponent}}
                $index := 0$\;
                $S :=$ \textit{empty stack}\;
                \ForEach{$v$ in $V$}{
                    \If{$v.index$ is undefined}{
                        $strongconnect(v)$\;
                    }
                }
                \caption{Tarjanův algoritmus}
            \end{algorithm}
\end{frame}

\section{Algoritmus 2}
\begin{frame}{Algoritmus 2. část}
    \begin{algorithm}[H]
        \DontPrintSemicolon
        \SetAlgoNoLine
        \SetAlgoNoEnd
        \SetKwProg{Fn}{function}{}{}

        \Fn{$strongconnect(v)$}{
            $v.index := index$\;
            $v.lowlink := index$\;
            $index := index + 1$\;
            $S.push(v)$\;
            $v.onStack := true$\;
            \ForEach{$(v, w)$ in $E$}{
                \If{$w.index$ is $undefined$}{
                    $strongconnect(w)$\;
                    $v.lowlink := min(v.lowlink, w.lowlink)$\;
                }
                \ElseIf{$w.onStack$}{
                    $v.lowlink := min(v.lowlink, w.index)$\;
                }
            }
            \If{$v.lowlink = v.index$}{
                \Repeat{$w \ne v$}{
                    $w := S.pop()$\;
                    $w.onStack := false$\;
                }
            }
        }

        \SetAlgoFuncName{name}{autoref name}
    \end{algorithm}
\end{frame}

\section{Příklad}
\begin{frame}{Příklad}
    \begin{itemize}


        \item \uncover<1->{
            \alt<9->{
                \alt<20->{
                    {Rozdělení podle silných komponent}}
                    {Aktualizace nejmenších časů objevení uzlů rodičů}}
                    {Procházení grafu pomocí DFS}}
    
    \end{itemize}
    \begin{center}
        \begin{tikzpicture}[scale=1.25]
            \draw (1,0) node[circle,draw,minimum size=0.7cm,fill=orange!40] (a) { };
            \draw (1,-1) node[circle,draw,minimum size=0.7cm,fill=orange!40] (b) {};
           \draw (1,-2) node[circle,draw,minimum size=0.7cm,fill=orange!40] (c) { };
            \draw (1,-3) node[circle,draw,minimum size=0.7cm,fill=orange!40] (d) { };
            \draw (0,-3) node[circle,draw,minimum size=0.7cm,fill=orange!40] (e) { };
            \draw (0,-2) node[circle,draw,minimum size=0.7cm,fill=orange!40] (f) { };
            \draw (2,-0.5) node[circle,draw,minimum size=0.7cm,fill=orange!40] (g) { };
            \draw (2,-2) node[circle,draw,minimum size=0.7cm,fill=orange!40] (h) { };
      
            

            \uncover<1->{
                \draw [-latex](a) -- (b);
                \draw [-latex] (b) -- (c);
                \draw [-latex](c) --  (d);
                \draw [-latex](d) --  (e);
                \draw [-latex](e) --  (f);
                \draw [-latex](f) --  (c);
                \draw [-latex](a) --  (g);
                \draw [-latex](g) --  (b);
                \draw [-latex](g) --  (h);
            }

            \uncover<2->{
                \draw (1,0) node[circle,draw,minimum size=0.7cm,fill=orange!80] (a) {1};
            }

            \uncover<3-13>{
                \draw [-{Stealth[red]}](a) -- (b);
                \draw [line width=1pt, draw=red](a) -- (b);
                \draw (1,-1) node[circle,draw,minimum size=0.7cm,fill=orange!80] (b) {2};
            }
            
            \uncover<4-12>{
                \draw [-{Stealth[red]}](b) -- (c);
                \draw [line width=1pt, draw=red](b) -- (c);
                \draw (1,-2) node[circle,draw,minimum size=0.7cm,fill=orange!80] (c) {3};
            }

            \uncover<5-11>{
                \draw [-{Stealth[red]}](c) -- (d);
                \draw [line width=1pt, draw=red](c) -- (d);
                \draw (1,-3) node[circle,draw,minimum size=0.7cm,fill=orange!80] (d) {4};
            }

            \uncover<6-10>{
                \draw [-{Stealth[red]}](d) -- (e);
                \draw [line width=1pt, draw=red](d) -- (e);
                \draw (0,-3) node[circle,draw,minimum size=0.7cm,fill=orange!80] (e) {5};
            }

            \uncover<7-9>{
                \draw [-{Stealth[red]}](e) -- (f);
                \draw [line width=1pt, draw=red](e) -- (f);
                \draw (0,-2) node[circle,draw,minimum size=0.7cm,fill=orange!80] (f) {6};
            }

            \uncover<8>{
                \draw [-{Stealth[red]}](f) -- (c);
                \draw [line width=1pt, draw=red](f) -- (c);
            }

            \uncover<9->{
                \draw (0,-2) node[inner sep=0,circle,draw,minimum size=0.7cm,fill=orange!120] (f) {6/3};
            }

            \uncover<10->{
                \draw (0,-3) node[inner sep=0,circle,draw,minimum size=0.7cm,fill=orange!120] (e) {5/3};
            }

            \uncover<11->{
                \draw (1,-3) node[inner sep=0,circle,draw,minimum size=0.7cm,fill=orange!120] (d) {4/3};
            }

            \uncover<12->{
                \draw (1,-2) node[inner sep=0,circle,draw,minimum size=0.7cm,fill=orange!120] (c) {3/3};
            }

            \uncover<13->{
                \draw (1,-1) node[inner sep=0,circle,draw,minimum size=0.7cm,fill=orange!120] (b) {2/2};
            }

            \uncover<14-18>{
                \draw [-{Stealth[red]}](a) -- (g);
                \draw [line width=1pt, draw=red](a) -- (g);
                \draw (2,-0.5) node[circle,draw,minimum size=0.7cm,fill=orange!80] (g) {7};
            }

            \uncover<15>{
                \draw [-{Stealth[red]}](g) -- (b);
                \draw [line width=1pt, draw=red](g) -- (b);
            }

            \uncover<16-17>{
                \draw [-{Stealth[red]}](g) -- (h);
                \draw [line width=1pt, draw=red](g) -- (h);
                \draw (2,-2) node[circle,draw,minimum size=0.7cm,fill=orange!80] (h) {8};
            }

            \uncover<17->{
                \draw (2,-2) node[inner sep=0,circle,draw,minimum size=0.7cm,fill=orange!120] (h) {8/8};
            }

            \uncover<18->{
                \draw (2,-0.5) node[inner sep=0,circle,draw,minimum size=0.7cm,fill=orange!120] (g) {7/7};
            }

            \uncover<19->{
                \draw (1,0) node[inner sep=0,circle,draw,minimum size=0.7cm,fill=orange!120] (a) {1/1};
            }

            \uncover<20>{
                \draw (1,0) node[inner sep=0,circle,draw,minimum size=0.7cm,fill=gray!30] (a) {1/1};
                \draw (1,-1) node[inner sep=0,circle,draw,minimum size=0.7cm,fill=blue!30] (b) {2/2};
                \draw (1,-2) node[inner sep=0,circle,draw,minimum size=0.7cm,fill=green!30] (c) {3/3};
                \draw (1,-3) node[inner sep=0,circle,draw,minimum size=0.7cm,fill=green!30] (d) {4/3};
                \draw (0,-3) node[inner sep=0,circle,draw,minimum size=0.7cm,fill=green!30] (e) {5/3};
                \draw (0,-2) node[inner sep=0,circle,draw,minimum size=0.7cm,fill=green!30] (f) {6/3};
                \draw (2,-0.5) node[inner sep=0,circle,draw,minimum size=0.7cm,fill=brown!30] (g) { 7/7};
                \draw (2,-2) node[inner sep=0,circle,draw,minimum size=0.7cm,fill=yellow!30] (h) { 8/8};
            }




        \end{tikzpicture}


    \end{center}
\end{frame}

\section{Složitost}
\begin{frame}{Složitost}
\begin{block}{Časová složitost}
    \begin{itemize}
        \item $O(V + E)$
        \item $V$ - počet vrcholů
        \item $E$ - počet hran
        \item Způsobeno použitím DFS
    \end{itemize}
\end{block}

\begin{block}{Prostorová složitost}
    \begin{itemize}
        \item $O(V)$
        \item $V$ - počet vrcholů
        \item Ukládání několika datových struktur
        \item Navštívené vrcholy, nejmenší čas objevení, \dots
    \end{itemize}
\end{block}
\end{frame}

\section{Využití}
\begin{frame}{Využití}
    \begin{itemize}
        \item Linearní čas. složitost - efektivní pro velké grafy
        \item Základ pro další grafové algoritmy
        \item Aplikovatelnost v analýze sítí - identifikace kritických komponent
        \item Optimalizace překladačů
    \end{itemize}
\end{frame}

\section{Reference}
\begin{frame}{Reference}
    \nocite{*}
    \bibliographystyle{czechiso}
	\renewcommand{\refname}{Reference}
	\bibliography{ref}

\end{frame}

\end{document}